\section{Lapskaus}
%Antall porsjoner
%Hvor lang tid det tar

\paragraph{Ingredienser}
\begin{itemize}[noitemsep]
	\item 2 pakker bacon i terninger
	\item smør til steking
	\item 2 poser lapskausgrønnsaker (trenger ikke å tines)
	\item 1 liten purre
	\item 1 kvist frisk timian eller 1 ts tørket timian
	\item 1 porsjonsbeger TORO oksefond
	\item 8 dl vann
	\item 1 pose potetmospulver
	\item salt og pepper
	\item flatbrød, til servering
\end{itemize}

\paragraph{Framgangsmåte}
\begin{enumerate}[noitemsep]
	\item Smelt litt smør i ei tykkbunnet, romslig gryte, og fres baconbitene på god varme i noen minutter.
	\item Hakk purren smått og ha den i med posegrønnsakene. La alt surre i smør/baconfett på middels varme et par minutter mens du rører.
	\item Hell over vann og fond, og la det koke i fem minutter.
	\item Fisk opp timiankvistene hvis du har brukt frisk timian.
	\item Synes du lapskausen er for tynn, rører du i litt potetmospulver slik at den tykner. Spe med litt vann hvis den er for tykk.
	\item Kvern over pepper, og smak til om det trengs mer salt. Serveres gjerne med flatbrød med smør.
\end{enumerate}

\paragraph{Tips:}
\begin{itemize}[noitemsep]
	\item Tilsett terninger med persillerot og pastinakk i grønnsaksblandingen.
	\item Bruk nok pepper! Gjerne litt mer enn du tror er nødvendig, det blir som regel alltid godt.
	\item Rund av smaken med litt røkt paprika.
	\item   Bytt ut baconet med chorizo som du freser i litt varm olje og krydrer med hvitløk og chili. Det vil gi en fin og gyllen lapskaus med et krydderkick.
\end{itemize}

Kilde:\url{https://coop.no/extra/tid--penger/rask-lapskaus-med-bacon} Coop Xtra -- Slik lager du god, gammeldags lapskaus på et kvarter
