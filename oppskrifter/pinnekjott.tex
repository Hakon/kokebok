\section{Pinnekjøtt}
Åtte porsjoner

\paragraph{Ingredienser}
\begin{itemize}[noitemsep]
	\item 3 kg kokefaste poteter blir ca 3 poteter per person
	\item 3 poteter ekstra til kålrabistappen
	\item 1 gulerot
	\item 4.5 kg kålrabi / ca 4 store
	\item 400 gram pinnekjøtt per pers (Handles hos meny eller Brakstads Etterfølgere (slakter på bystasjonen) eller hos slakteri i godvik)
	\item Bjørkepinner
	\item Evt desert
\end{itemize}

\paragraph{Framgangsmåte}
\begin{enumerate}[noitemsep]
	\item Skrell potetene og kok de, hvis de ikke skal kokes med en gang bør de ligges i en panen og tildekket med vann
	\item Kok 2--3 poteter ekstra til kålrabistappen, og en gulerot
	\item Skrell kålrabien,kutt i firkanter
	\item Ha litt salt i vannet og kok møre (30--45min)
	\item Når de er ferdigkokt: ha oppi en kokt gulrot og 2--3 poterer for en mildere smak
	\item Smak til med kremfløte og en liten ause pinnefett
	\item fyll en panne pannen med vann og ligg alle pinnene oppi over natten, skift vann om morgenen ellers blir vannet veldig salt
	\item La pinnekjøttet ligge i vann  til kl ca 14
	\item Ha bjørkepinner i bunnen av pannen til pinnekjøttet, (butikkpinner er greit, damprist går og), og  ha 2--3cm vann i bunnen
	\item Kok opp med lokket på, og skru ned så det bare akurat koker. Sørg for at man ikke tørrkoker, da blir det dårlig smak
	\item Kok/damp i tre timer, til kjøttet på pinnene løsner fra beinet
	\item Sjekk med jevne mellomrom (VIKTIG)
	\item Øs av pinnefettet fra toppen når pinnekjøttet er ferdig og ha det i en liten panne på varme, til maten er klar for å serveres
	\item Server på varme tallerkner
\end{enumerate}


\subsubsection{Notater}

\paragraph{Julebord hos Heine}
Åtte personer.
400gram pinnekjøtt * 8 personer = 3.2kg * 360kr/kg = 1150,-
500kr i kremfløte, vaniljestang og bringebær
50kr kålrabi
50kr i poteter
Totalt circa 1750,-/ 8 = 220,- per pers


Endte opp med:
500g Røykt Vossakjøt pinnekjøtt fra Meny /per person til 900.3 kr (359kr/kg)
4.5kg kålrot 44kr
3.1kg poteter 45kr
+ Pannacotta til alle
Totalt 150kr per pers
