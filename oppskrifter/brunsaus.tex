\section{Brun saus}


\subsection{Brun saus ala Pappa}

\paragraph{Ingredienser}
\begin{itemize}[noitemsep]
	\item Maizenajevner (mer enn det står på pakken)
	\item Seks osthøvelslicer med brunost
	\item 1 Toro buljongterning
	\item Idun sennep
\end{itemize}


\paragraph{Framgangsmåte}
\begin{enumerate}[noitemsep]
	\item Hiv oppi alt når det koker
	\item Smak det til
\end{enumerate}

Kilde: Pappa/Frode Lindseth


\subsection{Brun saus fra Enkel Kokebok}
\paragraph{Ingredienser}
\begin{itemize}[noitemsep]
	\item 4ss smør
	\item 5ss hvetemel
	\item 1 terning buljong
	\item salt og pepper
	\item Evt 1 løk
	\item Sukkerkulør / negro
\end{itemize}


\paragraph{Framgangsmåte}
\begin{enumerate}[noitemsep]
	\item Smelt og brun smøret og tilsett hvetemel. Brun over svak varme til blandingen får en nøttebrun farge.
	\item Spe med varm kraft eller buljong og rør godt mellom hver speing.
	\item La sausen trekke under lokk ca 10 minutter
	\item Smak til med salt og pepper. Tilsett eventuellt litt løk, og om ønskelig kan sausen farges med sukkerkulør. 1 ss solbærsaft og 1 ts soyasaus gir en god saus.
	\item Server sausen ved siden av kjøttkakene eller legg kjøttkakene oppi. Kjøttkakene kan etterkokes i sausen. Da får den en fyldigere smak.
\end{enumerate}

Kilde: Enkel Kokebok
