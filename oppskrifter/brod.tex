\section{Ernæringskorrekt brød}
Nok til 2 brød

\paragraph{Ingredienser}
\begin{itemize}[noitemsep]
	\item 1 l kaldt vann
	\item 12,5 gram fersk gjær (tilsvarer 1/4 pose tørrgjær)
	\item 10 gram flaksalt (for eksempel Maldon)
	\item 150 gram  linfrø
	\item 160 gram  havregryn
	\item 180 gram  siktet hvete
	\item 240 gram  sammalt hvete
	\item 240 gram  sammalt rug
\end{itemize}

\paragraph{Framgangsmåte}
\begin{enumerate}[noitemsep]
	\item Fremgangsmåte dag 1:
	\item Bland sammen ingrediensene og elt godt (rundt 20--30 minutter i kjøkkenmaskin) til deigen henger godt sammen
	\item La deigen stå fremme i rundt en time før du setter den i kjøleskap i minimum 12 timer
	\item Fremgangsmåte dag 2:
	\item Sett stekeovnen på 200 \degree C
	\item Fordel deigen i to brødformer smurt med olje
	\item Stek brødene i cirka en time og 15--30 minutter (steketiden avhenger av stekeovnen og størrelsen på brødformene), til brødene slipper formene
	\item Legg gjerne et bakepapir over brødene halvveis i steketiden
	\item Avkjøl brødene på rist og pakk dem inn i håndklær
	\item Ikke skjær i brødene før de er ordentlig avkjølt
\end{enumerate}

Kilde: Oppskriften ble utviklet av to tidligere studenter på samfunnsernæring ved Høgskolen i Oslo og Akershus, Kjersti Lilleberg og Line Jensen, som en del av deres masteroppgaver.

Det ideelle brødet oppfyller utvalgte kriterier knyttet til både helse, smak, lukt, utseende, miljø og praktiske faktorer, og oppskriften ble dessuten utviklet på bakgrunn av litteratursøk i vitenskapelige databaser.

Det `ideelle' brødet har en pris på omtrent 13 kroner per brød (inkludert strømutgifter). Å kjøpe et tilsvarende brød i butikken koster imidlertid langt mer.

Brødet har en grovhetsgrad på 78 prosent, og fyller derfor fire felt på Brødskalaen og oppnår betegnelsen «ekstra grovt». Det oppfyller også kriteriene for Nøkkelhullet. Sist, men ikke minst inneholder brødet lite kalorier, kun 173 kcal per 100 gram brød.
