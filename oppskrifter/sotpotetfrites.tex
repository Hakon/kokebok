\section{Søtpotetfrites}
\label{frites}

2 porsjoner
% Hvor lang tid det tar
Pris 35,-

\paragraph{Ingredienser}
\begin{itemize}[noitemsep]
	\item 500 gram søtpotet
	\item 1 ts Maizenna maisstivelse
	\item 2 ss olivenolje
	\item salt og pepper
	\item evt timian, oregano, chili
\end{itemize}

\paragraph{Framgangsmåte}
\begin{enumerate}[noitemsep]
	\item Skrell søtpoteten, og kutt den i jevnstore fries. Det er viktig at bitene har omlag samme størrelse, slik at alle bitene blir jevnt stekt.
	\item Legg de oppkuttede søtpotetene i en bolle med kaldt vann 10--15 minutter. Dette er med på å fjerne litt av stivelsen i potetene, ifølge matbloggeren We Are Not Foodies
	\item Hell av vannet, og tørk godt av søtpotetene. Legg dem deretter i en plastikkpose med en halv teskje Maizenna, lukk posen, og rist godt så alle potetbitene er jevnt dekket av et tynt lag maisstivelse. Tanken her er at laget med maisstivelse blir sprøtt, skriver We Are Not Foodies
	\item Her er det viktig å ikke bruke for mye maisstivelse, da dette kan føre til at potetene kun smaker maisstivelse etter stekingen
	\item Åpne posen, og tilsett cirka én spiseskje olivenolje, samt ønsket krydder. Lukk posen, og rist godt så søtpotetfries-en er ordentlig dekket.
	\item Dekk et stekebrett med bakepapir, og legg søtpotetene jevnt utover. Pass på at de ikke dekker hverandre, da dette kan føre til bløte og slappe biter
	\item Sett stekebrettet inn i ovnen på 225 grader varmluft. Når bitetene begynner å se godt steikt ut må man ta brettet ut og snu bitene så alle sider blir tilstrekkelig stekt.
	\item Sett dem inn igjen i 10 nye minutter. Etter dette må man sette ovnsdøren på gløtt, og la de steke i 5--10 minutter til de har fått en sprø overflate, og et mykt og seigt indre.

\end{enumerate}

Tips:
Kilde:\href{https://coop.no/extra/mat--trender/garantert-spro-sotpotetfries/}{Cop xtra - søtpotet pommes frites}
